\chapter{Einführung}
\label{sec:einfuehrung}

\section{Aufbau}
Dieses Kapitel stellt den Autoren vor und gibt eine Einführung in das Thema.
Die folgenden Passagen stellen danach das Gebiet des Social Engineering und Phishing vor.
Zum Schluss gibt es noch ein Abschlusswort.

\section{Über den Autor}
\label{sec:einfuehrung:autor}
Mein Name ist Simon Lang. Als gelernter Informatiker arbeite ich seit 2006 in der Webentwicklungsbranche.
Seit 2012 studiere ich Informatik an der \Gls{zhawLabel}.
Da in der Webentwicklung die Sicherheit sehr wichtig ist habe ich das Fach Informationssicherheit und Kryptografie mit der Vertiefung Sicherheitsanwendungen und Public Key Infrastructure gewählt. 
Phishing und Social Engineering arbeitet stark mit dem Internet zusammen weshalb dieses Thema für die Arbeit gewählt wurde.

\section{Über dieses Dokument}
In dieser Arbeit werden die beiden Themen Social Engineering und Phishing vorgestellt. 
Das Dokument soll auch von Lesern ausserhalb der Informatik verstanden werden, obwohl grundlegende Kenntnisse von Computern und dem Internet vorausgesetzt werden.

\section{Phishing und Social Engineering}
Sicherheit ist ein relativer Begriff. Hacker und Sicherheitsexperten liefern sich einen stetigen Kampf. Die eine Seite versucht durch Angriff oder Viren in geschützte Netzwerke einzudringen, die andere Seite versucht dies zu verhindern.
Sicherheit ist dabei nur die Schwierigkeit um einen Angriff erfolgreich durchzuführen. Denn einen Weg um in einen geschützten Bereich einzudringen gibt es immer.
Zu Beginn der EDV Ära war ein unbefugtes Eindringen zum Teil sehr einfach. Das Sicherheitsbewusstsein von Personen und Firmen wurde jedoch immer höher, und so wurde auch das Hacken immer schwieriger. Deshalb suchten Angreifer andere Wege für einen Einbruch. 
Maschinen machen keine Fehler. Sind sie sicher Programmiert ist ein Hack schwierig. Dagegen ist das irren Menschlich. Diese Tatsache versucht man sich beim Social Engineering und Phishing zu seinem Gunsten zu nutzen.
Beim Social Engineering tritt der Hacker aus dem Keller hervor und tritt mit dem Angriffziel, einem Benutzer, Administrator, etc., direkt in Kontakt. Es wird versucht durch einen Fehler des Menschen eine Sicherheitslücke zu finden welche Angegriffen werden kann. 
Das Phishing ist eine Unterkategorie des Social Engineering. Durch gefälschte E-Mails oder Kurznachrichten wird das Opfer meistens auf eine Webseite geleitet wo versucht wird persönliche Informationen zu erschleichen.

