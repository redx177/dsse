\chapter{Software Maintenance}
Eine Webseite kann von einer simplen \Gls{glos:htmlLabel} Datei bis zu einem komplexen Konstrukt mit mehreren Webservern und Datenbanken reichen. Deshalb ist die Pflege sehr abhängig von der Grösse der Seite. 

\section{Service Layer Agreements}
Zum Beispiel bei Internetshops, Reiseportalen, etc. ist es von existenzieller Bedeutung für die Firma, dass die Seite läuft. Durch einen Ausfall können schnell Millionen von Franken Umsatz verloren gehen. Bei diesen Fällen werden \glspl{slaLabel} eingesetzt. Diese Regeln, wie lange die Seite im Jahr Online bleiben muss. Der Wert liegt oft bei 99.99\% oder noch höher. Es werden die Anzahl an 9'en gezählt. Wenn man in einem SLA von vier 9’en spricht, muss die Seite demnach 99.9999\% online sein. Dies entspricht 31 Sekunden im Jahr.

Über \glspl{slaLabel} wird zusätzlich die Reaktionszeiten, Fehlerbehebung oder Weiterentwicklungen definiert. Es kann abgemacht werden, dass die Seite innerhalb von zwei Sekunden eine Antwort liefert, dass Fehler binnen zwei Tagen behoben oder eine Weiterentwicklung maximal eine Woche dauern darf.

\section{Service Desk}
Zur Maintenance gehört auch der Support für die Benutzer. Bei Fragen muss es eine Anlaufstelle geben an welche sich die betroffenen melden können.

Es gibt drei Arten von Service Desks:
\begin{itemize}
\item Interne
\item Externe
\item Virtuelle
\end{itemize}

Das interne Service Desk wird von der Firma selber betrieben. Für das externe wird eine eigene Firma mit dem Auftrag engagiert. Für das virtuelle Service Desk gibt es auch den Ausdruck \textit{Follow The Sun}. Dazu gibt es eine Anlaufstelle für alle Fragen. Je nach Tageszeit erhält jedoch immer ein anderes Service Desk die Anfrage, Je nachdem welches gerade aktiv ist. Weltweit gibt es somit drei bis vier Instanzen. Diese Möglichkeit eignet sich für internationale Firmen und spart kosten, da keine Nachtzuschläge auf die Löhne entrichtet werden müssen.